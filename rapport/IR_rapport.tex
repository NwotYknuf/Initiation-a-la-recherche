\documentclass[soumission]{ir}

\titre{Instructions aux auteurs pour la préparation d'articles sous \LaTeX pour le projet IR}

\auteur{Alexandre Blansché\affil{1} et Dominique Michel\affil{2}}

\affiliation{
    \affil{1}Université de Lorraine\\
    UFR MIM -- LORIA\\
    3 Rue Augustin Fresnel, 57070 Metz\\
    alexandre.blansche@univ-lorraine.fr\\
    %
    \affil{2}Université de Lorraine\\
    UFR MIM -- LGIPM\\
    3 Rue Augustin Fresnel, 57070 Metz\\
    dominique.michel@univ-lorraine.fr
}

\titrecourt{Titre court de votre article en 10 mots maximum}

\nomcourt{A. Blansché et D. Michel}

\encadrant{Nom de l'encadrant}

\resume{Ces instructions sur la préparation d'articles pour le projet IR sous \LaTeX\ doivent être respectées strictement pour assurer une présentation cohérente aux articles. Merci de ne pas modifier le formatage des différents textes. Ce résumé doit faire au plus 150 mots.}

\motcle{\LaTeX{}, Initiation à la recherche, Avoir une bonne note}

\begin{document}

\section{Introduction}

\section{Consignes générales}\label{sec:consignes}

\subsection{Section de niveau 2}

\subsubsection{Section de niveau 3}

\subsection{Liste d'éléments}

\cite{goldberg89genetic}


\subsection{Notes en bas de page}

\section{Références bibliographiques et citation}\label{sec:biblio}

\section{Conclusion}

\remerciement{Un court paragraphe peut-être utilisé à la fin pour les remerciements.}

\bibliographystyle{apalike}
\bibliography{IR_biblio}

\appendix

\section*{Annexe}

Voici un exemple d'annexe. S'il y a plusieurs annexes, on utilisera la commande \texttt{$\backslash$section\{Nom de l'annexe\}}.

\end{document}
